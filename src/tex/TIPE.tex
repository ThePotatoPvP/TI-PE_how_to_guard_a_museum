\documentclass[12pt]{article}
\usepackage[T1]{fontenc}
\usepackage[utf8x]{inputenc}
\usepackage{amsmath,amsthm,amsfonts,amssymb}
\usepackage{graphicx}
\usepackage[shortlabels]{enumitem}
\usepackage{accents}
\usepackage{stmaryrd}
\usepackage[margin=25mm]{geometry}
\usepackage[french]{babel}



\setlist[enumerate,1]{label=\arabic*)}
\setlist[enumerate,2]{label=\alph*)}


\date{\today}
\author{Corentin Sallin}
\title{TIPE}



\renewcommand{\theenumi}{\Alph{enumi}}

\newcommand{\N}{\mathbb{N}}
\newcommand{\Z}{\mathbb{Z}}
\newcommand{\R}{\mathbb{R}}
\newcommand{\Q}{\mathbb{Q}}
\newcommand{\C}{\mathbb{C}}
\newcommand{\K}{\mathbb{K}}
\newcommand{\primes}{\mathbb{P}}
\renewcommand{\L}{\mathcal{L}}
\newcommand{\M}{\mathcal{M}}
\newcommand{\T}{\mathcal{T}}
\newcommand{\Diag}{\mathcal{D}}
\newcommand{\curlyv}{\mathcal{V}}


\newcommand{\CC}{\mathcal{C}}
\newcommand{\Int}{\mathrm{Int}}
\newcommand{\id}{\mathrm{id}}
\newcommand{\eps}{\varepsilon}
\newcommand{\mnr}{\mathcal{M}_n(\R)}
\newcommand{\mnk}{\mathcal{M}_n(\K)}
\newcommand{\glnr}{\mathrm{GL}_n(\R)}
\newcommand{\glnk}{\mathrm{GL}_n(\K)}
\newcommand{\ps}{\mathcal{P}} % Power set

\newcommand{\D}{\mathop{}\!\mathrm{d}}
\newcommand{\ds}{\displaystyle}
\newcommand*{\ensemble}[3][]{#1\{ #2 \mid #3 #1\}}
\newcommand{\ab}{\mathopen{]}a,b\mathclose{[}}
\newcommand{\Vvert}{\vert\kern-0.25ex\vert\kern-0.25ex\vert}
\newcommand{\tnorm}[1]{\Vvert #1 \Vvert}
\newcommand{\<}{\langle}
\renewcommand{\>}{\rangle}

\DeclareMathOperator{\Ker}{Ker}
\DeclareMathOperator{\Isom}{Isom}
\DeclareMathOperator{\Tr}{Tr}


\theoremstyle{definition}
\newtheorem{exo}{Exercice}
\newtheorem{q}{Question}[exo]

\newtheorem{theorem}{Théorème}
\newtheorem{lemma}[theorem]{Lemme}
\newtheorem{definition}[theorem]{Définition}
\newtheorem*{corollary}{Corollaire}

\newtheorem*{proposition}{Proposition}
\newtheorem*{remark}{Remarque}
\newtheorem*{example}{Exemple}

\begin{document}

\begin{titlepage} %Personnaliser la page de présentation
    \begin{center}
        
            \textsc{ \LARGE{\textbf{T}ravail d'\textbf{I}nitiative \textbf{P}ersonnel \textbf{E}ncadré\\}}
        \textnormal{\LARGE{CPGE MP2I-MPI\\}}
        \vspace{6.5cm}
        \textsl{\Huge{Problème du collectionneur d'art et extension au cadre
        de la surveillance d'une ville}}\\
        \vspace{6cm}
        \textnormal{\Large{\bf Corentin SALLIN\\}}
    \end{center}
    \footnotetext{Problème original posé par V. Klee en 1973}
\end{titlepage}

\section*{Problème}

On se donne un polygône à  \(n\) sommets, quel est le nombre minimum de points 
à partir desquels on peut voir tout point de l'intérieur du polygône.

Le cas du musée peut s'étendre à celui de la surveillance d'une ville en considérant 
des polygônes avec des trous, coupant la visibilité du garde.


\section*{Liste de choses à faire}

\begin{enumerate}[\(\hookrightarrow\)]
    \item Se familiariser avec le vocabulaire utilisé dans les documents
    \item Etude de la décomposition de polygônes \cite{Lubiw,star,Chatal}
    \item Etude des différentes preuves du théorème de Chvátal \cite{Fisk}
    \item Etude des algorithmes déjà existants répondant au problème \cite{Aggarwal,star,Couto}
    \item Implémentation de ces algorithmes, tests avec étude statistique.
    \item Extension au modèle d'une ville, un polygône ayant des trous,
    vaguement abordé dans \cite{Aggarwal,Aldo}
\end{enumerate}

\begin{theorem}(de Chvátal)
    Soit \(S\) un polygône à \(n\) sommets, il existe un ensemble \(T\) d'au 
    plus \(\left\lfloor \dfrac{n}{3} \right\rfloor\) points de \(S\) tel que 
    pour tout point \(p \in S\) il existe un point \(q \in T\) tel que le segment 
    \(pq\) est compris dans \(S\).
    
\end{theorem}

\begin{center}
    Ce problème est équivalent à 3-SAT \cite{Aldo,Krohn}
\end{center}

Mon étude se porte surtout sur le cas du polygône avec des trous, plus 
prôche du cadre de la ville. Il faut donc différencier le cas des gardes de mur 
et des gardes intérieurs comme cités dans \cite{Aldo}. Je chercherai ici à surveiller 
tout l'intérieur est non seulement les parois comme on pourrait le faire dans 
un musée.

\begin{thebibliography}{100}
    
    \addtolength{\leftmargin}{0.2in} % Pour aligner avec la ligne suivante
    \setlength{\itemindent}{-0.2in}

    \bibitem{Aggarwal} A. Aggrawal, \textsl{The art gallery theorem: It's variations,
    applications, and algorithmic speed}, thèse, Université Johns Hopkins, 1984

    \bibitem{Aldo} A. Laurentini, \textsl{Guarding the walls of an art gallery},
    pp 265-278, 1999

    \bibitem{Lubiw} A. Lubiw, "Decomposing polygonal regions into convex 
    quarilaterals" \textsl{Proc 1st ACM Symposium on Computational Geometry},
    pp 97-106, 1985

    \bibitem{star} G. T. Toussaint, D. Avis, "An efficient algorithm for 
    decomposing a polygon into star-shaped polygon"\textsl{Pattern Recognition}, 
    pp 395-398, 1981

    \bibitem{Krohn} Krohn, A. Erik, J. Bengt "Approximate guarding of 
    monotone and rectilinear polygon" \textsl{Algorithmica}, pp 564-594, 2013

    \bibitem{Divine} M. Aigner, G. M. Ziegler, \textsl{Proofs from THE BOOK}  
    (4th ed), pp 231-234, Chapter 35, 2009

    \bibitem{Couto} M. Couto, P. de Rezende, C. de Souza, \textsl{An exact algorithm
    for minimizing vertex guards on art galleries}, pp 425-448, 2011

    \bibitem{valtr} P. Valtr, "Guarding galleries where no point sees a small 
    area" \textsl{Israel Journal of Mathematics}, pp 1-16, 1998
    
    \bibitem{Fisk} S. Fisk, "A short proof of Chvátal's watchman theorem" 
    \textsl{Journal of Combinatorial Thoery, Series B}, pp 324, 1978

    \bibitem{Chatal} V. Chvátal, "A combinatorial theorem in plane geometry" 
    \textsl{Journal of Combinatorial geometry, Series B}, pp 39-41, 1975

\end{thebibliography}

\end{document}
